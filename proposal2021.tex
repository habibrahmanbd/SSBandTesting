\documentclass[12pt]{article}

% This first part of the file is called the PREAMBLE. It includes
% customizations and command definitions. The preamble is everything
% between \documentclass and \begin{document}.

\usepackage[margin=1in]{geometry}  % set the margins to 1in on all sides
\usepackage{graphicx}              % to include figures
\usepackage{epstopdf}
\usepackage{amsmath}               % great math stuff
\usepackage{amsfonts}              % for blackboard bold, etc
\usepackage{amsthm}                % better theorem environments
\usepackage{natbib}



\begin{document}

\begin{center}
{\bf \Large Automatic Construction of WordNets}  \\
\vspace{.1in}
{\em Bradley Hauer and Grzegorz Kondrak}
\end{center}


\paragraph{Problem}

The problem we seek to address is the automatic construction of WordNets. 
A WordNet is a rich semantic and lexicographic resource 
which groups words into synonym sets, or synsets, 
and which records relations between synsets, such as hypernymy and hyponymy. 
WordNets have proven to be extremely useful 
in natural language processing (NLP), machine learning, 
and computational linguistics.

Construction of a WordNet typically involves 
a large amount of manual work by language experts, 
which often results in limited coverage and accuracy, 
especially for lower-resourced languages. 
Maintaining or expanding a WordNet 
requires a further investment of time and resources. 
This poses a problem 
for the development and application of WordNet-based methods. 
We intend to address this problem by developing and applying methods 
to automatically construct a WordNet 
for a given language from multilingual corpora and thesauri.
As part of our research, 
we will also develop methods to automatically validate existing WordNets.


\paragraph{Related Work}

Our goal of constructing a WordNet using multilingual data
builds upon prior work, beginning with \cite{dyvik2004},
who used translations as a ``semantic mirrors'' to construct synsets.
\cite{sagot2008} construct a French WordNet
using multilingual resources,
but show that there is substantial room for improvement 
in the process, as precision and recall compared to a manually constructed
WordNet are less than 80\%.
\cite{lohk2018}, inspired by Dyvik's method,
use a neural machine translation system to induce synsets.


\paragraph{Importance of the Problem}

WordNets have been applied to a variety of semantics tasks, including
sentiment analysis, relation extraction, and word sense disambiguation (WSD).
State-of-the-art systems for these tasks 
often depend on a WordNet as a resource,
especially in unsupervised settings.
However, WordNets are language-specific,
precluding the application of many state-of-the-art methods
to languages with fewer available resources.
A language-independent method
for WordNet construction
would therefore greatly increase the number of languages
covered by modern NLP systems.

WordNets may be of greater use to language learners than traditional
dictionaries.
There is a substantial body of prior work on cross-lingual
alignment of WordNets % for example, \cite{rudnicka2012},
which could naturally be extended to align the WordNets produced
using our method.
To goal of such alignment is to produce a mapping which,
to the extent possible, maps synsets from a WordNet in one language
to a semantically equivalent synset for a WordNet in another language.
Such alignment creates a sense-level bilingual lexicon;
since different senses of a word are 
often lexicalized differently in other languages.



% \vspace{.1in}
% \paragraph{Conclusion}


\newpage
\noindent
\paragraph{Acknowledgements}
This proposal was prepared through discussions 
with Prof. \ldots and PhD student \ldots. 

\bibliographystyle{plainnat}

\bibliography{proposal2021}


\end{document}
